\documentclass[12pt]{article}
\usepackage{amsmath,amssymb}
\usepackage[margin=3cm]{geometry}
\title{Ranking System}
\author{Peter Schmidt-Nielsen}
\begin{document}
\maketitle

\section{Two player games}
Our goal is to design a ranking system for team games.
We proceed in a similar way to how Microsoft Research's TrueSkill is designed.
First, we explore the version for two player games.

Player $i$ is assumed to have a performace $p_i$ in the game sampled from $\mathcal{N}_{\mu_i, \sigma_i}$, with
$\mathcal{N}$ being the normal, whose PDF is of course
\[ \mathcal{N}_{\mu, \sigma}(x) = \frac1{\sigma \sqrt{2 \pi}} e^{-\frac{(x-\mu)^2}{2\sigma^2}}. \]
We will use $\mathcal{N}_i$ as shorthand for $\mathcal{N}_{\mu_i, \sigma_i}$.

Given two players, it is assumed that a player wins if his or her performance exceeds that of his or her opponent.
A player who is won against is a loser, and if no player wins both players are said to draw.
Thus, drawing is a measure zero event (our first major modeling error).

\subsection{Figures of interest}
There are three main figures we care to compute, in a two player game.
\begin{enumerate}
\item
	Probability of w.l.o.g. player one winning $\mathbb{P}(p_1 > p_2)$.
	We care about this so we can compute how much of an upset it is if the underdog wins.
\item
	Maximum likelihood estimate of $p_1$, assuming player one won.
	That is:
	\[ \argmax_{x} \mathbb{P}(p_1 = x\ |\ p_1 > p_2) \]
	This is of obvious interest -- it tells us how well a player likely did given that he or she beat another player.
\item
	The expectation of $p_1$ given that player one won, namely $\mathbb{E}[p_1\ |\ p_1 > p_2]$.
	We will use this value to update a player's mean performance after a round.
\end{enumerate}

\subsubsection{Probability of a win}
Given two players, with parameters $(\mu_1, \sigma_1)$ and $(\mu_2, \sigma_2)$ respectively, we seek the probability that $p_1 > p_2$.
Observe that this holds exactly when $p_2 - p_1 < 0$, or namely when a sample from $p_2 - p_1 \sim \mathcal{N}_{\mu_2 - \mu_1,\sqrt{\sigma_1^2 + \sigma_2^2}}$ is negative.
This is, of course, equal to its CDF evaluated at zero:
\[ \mathbb{P}(p_1 > p_2) = \Phi\left(\frac{\mu_1 - \mu_2}{\sqrt{\sigma_1^2 + \sigma_2^2}}\right) \]

\subsubsection{MLE performances}
Given that player one won, we want to know what his or her performance most likely was.
To do this, we do a Maximum Likelihood Estimation of $p_1$ given the four player parameters and the fact that $p_1 > p_2$.
We compute the likelihood as follows:
\begin{align*}
\mathbb{P}(p_1 = x\ |\ p_1 > p_2) &= \frac{\mathbb{P}(p_1 > p_2\ |\ p_1 = x) \mathbb{P}(p_1 = x)}{\mathbb{P}(p_1 > p_2)} \\
&\propto \mathbb{P}(p_1 > p_2\ |\ p_1 = x) \mathbb{P}(p_1 = x) \\
&= \int_{-\infty}^x \mathcal{N}_2(x') \, \mathrm{d}x' \cdot \mathcal{N}_1(x)
\end{align*}
We seek to maximize this quantity, which we'll call $L(x)$.
So we compute $L'(x)$:
\begin{align*}
L(x) &= \int_{-\infty}^x \mathcal{N}_2(x') \, \mathrm{d}x' \cdot \mathcal{N}_1(x) \\
L'(x) &= \partial_x \left[ \int_{-\infty}^x \mathcal{N}_2(x') \, \mathrm{d}x' \cdot \mathcal{N}_1(x) \right] \\
&= \mathcal{N}_2(x) \mathcal{N}_1(x) +
\int_{-\infty}^x \mathcal{N}_2(x') \, \mathrm{d}x' \cdot \mathcal{N}'_1(x)
\end{align*}
Note that:
\[
	\int_{-\infty}^x \mathcal{N}_{\mu, \sigma}(x') \, \mathrm{d}x' = \Phi\left(\frac{x - \mu}{\sigma}\right)
	\qquad \text{and} \qquad
	\mathcal{N}'_{\mu,\sigma} = - \frac{x - \mu}{\sigma^2} \mathcal{N}_{\mu, \sigma}(x)
\]
Thus: 
\[
	L'(x) =
	\mathcal{N}_2(x) \mathcal{N}_1(x) -
	\Phi\left(\frac{x - \mu_2}{\sigma_2}\right) \frac{x - \mu_1}{\sigma_1^2} \mathcal{N}_1(x)
\]
Therefore, to extremize $L$ we can set $L' = 0$, and reduce to:
\[ \mathcal{N}_2(x) = \Phi\left(\frac{x - \mu_2}{\sigma_2}\right) \frac{x - \mu_1}{\sigma_1^2} \]
This is readily solvable analytically, so we resort to using Newton's method to hone in on a solution.
TODO: Fill in the rest of what I've worked out.

\subsubsection{Expectation of performance}
Given that player one won, we could also wish to ask the expectation of $p_1$.
We can compute this as:
\begin{align*}
\mathbb{E}[p_1\ |\ p_1 > p_2] &= \frac1{\mathbb{P}(p_1 > p2)} \int_{-\infty}^\infty x L(x) \, \mathrm{d}x \\
&= \frac1{\sigma_1 \sqrt{2\pi}} \Phi\left(\frac{\mu_1 - \mu_2}{\sqrt{\sigma_1^2 + \sigma_2^2}}\right)^{-1}
   \int_{-\infty}^\infty x \Phi\left(\frac{x - \mu_2}{\sigma_2}\right) e^{-\frac{(x - \mu_1)^2}{2 \sigma_1^2}} \, \mathrm{d}x \\
%&= \mu_1 + \Phi\left(\frac{\mu_1 - \mu_2}{\sqrt{\sigma_1^2 + \sigma_2^2}}\right)^{-1}
%   \frac{\sigma_1^2}{\sqrt{2\pi (\sigma_1^2 + \sigma_2^2)}} e^{-\frac{(\mu_1 - \mu_2)^2}{2 (\sigma_1^2 + \sigma_2^2)}}
&= \mu_1 + \Phi\left(\frac{\mu_1 - \mu_2}{\sqrt{\sigma_s^2}}\right)^{-1}
   \frac{\sigma_1^2}{\sqrt{2\pi \sigma_s^2}} e^{-\frac{(\mu_1 - \mu_2)^2}{2 \sigma_s^2}} \\
&= \mu_1 + 2 \operatorname{erfc}\left(\frac{\mu_2 - \mu_1}{\sqrt{2 \sigma_s^2}}\right)^{-1}
   \frac{\sigma_1^2}{\sqrt{2\pi \sigma_s^2}} e^{-\frac{(\mu_1 - \mu_2)^2}{2 \sigma_s^2}}
\end{align*}
Where $\sigma_s^2 = \sigma_1^2 + \sigma_2^2$ is the variance of the sum (equivalently, and perhaps more saliently, difference) of the two normals.

\subsection{Putting it all together}
TODO: Write this up.

\section{Acknowledgements}
The idea to model performance of a player as draws from a normal variate came from Microsoft Research's TrueSkill, and likewise for the idea to model the performance of a team as the sum of the performances of the players.
Similarly, the decision to be very Bayesian about the whole thing was inspired by TrueSkill.

\end{document}
